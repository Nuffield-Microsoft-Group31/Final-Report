\documentclass[12pt]{article}
\usepackage{graphicx}
\usepackage{subfig}
\usepackage{float}
\usepackage[margin=1in,includefoot]{geometry}
\usepackage{fancyhdr}
\pagestyle{fancy}
\renewcommand{\footrulewidth}{1pt}

\begin{document}

\begin{titlepage}
\begin{center}
\begin{figure}[t]
\hspace*{0.35cm}\includegraphics[width=1.0\textwidth]{uclLogo}\\
\end{figure}
\line(1,0){300}\\
[0.25in]
\huge{\bfseries Nuffield Health \& Microsoft}\\
[2mm]
\line(1,0){200}\\
[1.5cm]
\textsc{\LARGE University College London}\\
\textsc{\normalsize Department of Computer Science}\\
\textsc{\normalsize COMP204P Systems Engineering}\\
\textsc{\normalsize Group 31}\\
[5cm]
\end{center}

\end{titlepage}
\tableofcontents

\newpage
\section{Abstract}\label{sec:abstract}





\newpage
\section{Context}\label{sec:context}

\subsection{Background}

\subsection{Purpose}\label{sec:purpose}

\par


\subsubsection{whatever1}


\subsubsection{whatever2}


\subsubsection{whatever3}



\newpage
\section{Team Member Summary}

\subsection{Mohammad Hossein Afsharmoqaddam}


\subsection{Jas Semrl}


\subsection{Marc de Fontenay}





\newpage
\section{Work Plan (GANTT CHART)}
To be as efficient and productive as possible and a Gantt Chart offers many advantages we decided to use this method to enhance our communication, and to be able to see our project over the long term and to track the results of each task. 
\par


\end{document}
